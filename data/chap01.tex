% !TeX root = ../__ylc_main.tex

\chapter{引言}

% 一篇学位论文的引言大致包含如下几个部分:
% 1、问题的提出;
% 2、选题背景及意义;
% 3、文献综述;
% 4、研究方法;
% 5、论文结构安排。
% \begin{itemize}
%   \item 问题的提出:要清晰地阐述所要研究的问题“是什么”。
%     \footnote{选题时切记要有“问题意识”,不要选不是问题的问题来研究。}
%   \item 选题背景及意义:论述清楚为什么选择这个题目来研究,即阐述该研究对学科发展的贡献、对国计民生的理论与现实意义等。
%   \item 文献综述:对本研究主题范围内的文献进行详尽的综合述评,“述”的同时一定要有“评”,指出现有研究状态,仍存在哪些尚待解决的问题,讲出自己的研究有哪些探索性内容。
%   \item 研究方法:讲清论文所使用的学术研究方法。
%   \item 论文结构安排:介绍本论文的写作结构安排。
% \end{itemize}

\section{研究背景}

近年来,随着计算机科学和人工智能领域的迅猛发展,大语言模型(Large Language Models, LLM)取得了显著的进步。这些模型,尤其是OpenAI的GPT-4及其后续版本,展示了在自然语言理解和生成方面的卓越能力。LLM不仅能进行复杂的语言任务处理,还在逻辑推理、数学计算等方面展现出了强大的潜力。这些技术的进步为各个领域带来了新的机遇和挑战,教育领域亦不例外。

然而,与科技的快速发展相比,传统教育实践的发展显得相对滞后。尤其是在作业批改和学生能力评估方面,传统的手工批改方式依然占据主导地位。比如对于高中数学解答题的批改,教师需要阅读学生的解答、判断每一步是否正确,这本身就已经需要较大的工作量了。但是要想针对学生的具体思路进行个性化的分析和指导,判断每一步是否符合标答还不够。对于一个最终答案错误的解答过程,我们希望得知学生具体的错误点在哪里;中途的推导具体哪一步是合理的、哪一步有错误;涉及到哪些学生没有掌握的知识点、方法或能力。要想精确地定位和解决问题,就要去挖掘每一个学生解题过程的思路细节,但是如果这些工作对于中学教师来讲更加繁琐和复杂,几乎不可能对每个学生付诸实践。

根据本研究过程中对普通高中数学教师的访谈,每个老师平均每天花在批改作业上的时间大约是两到三个小时。传统的方法不仅耗费大量时间和人力,而且难以深入了解和追踪每个学生在解题过程中的思路。教师在批改过程中往往只能看到学生的最终答案,而无法详细了解学生在解题过程中所经历的思维路径和所犯的具体错误。这种评估方式的局限性显然无法满足现代教育对个性化教学和精细化评估的需求。

LLM的出现,为解决这一问题提供了新的方法与可能性。通过使用几十到几百TB量级的文本作为预训练数据,语言模型拥有丰富的知识库和语义理解、生成能力。考虑目前综合能力最强的模型:OpenAI的GPT-4,在49.7\% 的情况下,GPT-4 已经能够通过图灵测试,让被试从语言风格与社会情感特征等方面认为其输出和正常人类的水平相符\cite{jones2024does}。在数学能力方面 GPT-4 的表现也非常优秀,在数学认知推理能力方面能够达到人类学生的中位数水平;尤其在等式与不等式、概率统计、排列组合等内容上的表现能够超越人类学生的平均表现\cite{zhuang2023efficiently}。

LLM在批改数学作业方面展现出了巨大的潜力。借助LLM的强大能力,我们可以自动化地解析和跟踪学生的解题过程,从而实现对学生思维能力的细致评估。LLM在语言处理方面的卓越表现,使其不仅能理解学生的文字表达,还能分析其解题步骤和逻辑思路。此外,LLM在数学计算和逻辑推理方面的能力,使其能够对学生的解题过程进行准确的评估和错因分析。


\section{问题提出}

本研究提出了一种基于LLM的学生思维能力评估方法,通过对解题过程中思路的详细建模,从而评估学生的思维能力。在思路建模的基础上,本研究还实现了解题过程的错因分析,对于学生可以准确地指出发生错误的步骤,对于教师可以获得班级学生的错因统计。本研究是将LLM应用于教育领域来进行思维建模的首次尝试,实验结果表明,该方法可以有效地评估学生的思维能力,为自动化题目讲解以及在中学的大规模应用提供了基础。


\section{研究意义}
本研究通过探索大语言模型(LLM)在教育领域中的应用,特别是在学生思维能力评估和作业批改中的潜力,具有多方面的重要意义。

1. 提高教育效率

首先,本研究旨在利用LLM强大的自然语言处理和逻辑推理能力,自动化地完成作业批改和错因分析,极大地减轻教师的工作负担。传统的作业批改方式不仅耗时费力,还难以实现对学生解题思路的深入分析。而通过LLM的应用,可以高效、准确地完成这一任务,从而释放教师更多的时间用于教学设计和个性化辅导。

2. 实现个性化教学

其次,本研究为个性化教学提供了新的工具和方法。通过对学生解题过程的详细建模和错因分析,LLM可以为每个学生提供针对性的反馈和指导。教师可以根据这些分析结果,更好地了解学生的思维模式和知识掌握情况,从而制定更加个性化的教学策略,帮助学生在其薄弱环节上取得进步。这种精细化的评估和指导,有助于提升学生的学习效果和学习体验。

3. 促进教育公平

此外,LLM的应用有助于促进教育公平。传统教育资源的分布不均,使得一些学生无法获得足够的个性化指导。而通过LLM的自动化分析和评估,可以为更多学生提供高质量的学习支持,尤其是在教育资源相对匮乏的地区。这样一来,可以有效地缩小教育差距,让更多学生受益于先进的教育技术。

4. 推动教育技术创新

本研究在教育技术领域的创新应用,具有重要的理论和实践价值。作为将LLM应用于学生思维建模和评估的首次尝试,本研究不仅丰富了LLM的应用场景,也为教育技术的进一步发展提供了有益的参考和启示。研究结果显示,LLM在理解和分析学生解题思路方面具有显著的优势,为未来教育技术的创新和应用奠定了坚实基础。

5. 提供大规模教育数据分析的可能性

最后,本研究还为大规模教育数据的分析和利用提供了可能性。通过LLM对学生解题过程的自动化建模和分析,可以收集和处理大量的教育数据。这些数据不仅可以用于学生个人和班级整体的学习情况分析,还可以为教育研究提供丰富的数据支持,帮助教育研究者深入探讨学生的学习行为和思维模式,从而推动教育理论的发展。

综上所述,本研究不仅在理论上为LLM在教育领域中的应用提供了新的视角和方法,在实践上也展示了LLM在提高教育效率、实现个性化教学和促进教育公平等方面的巨大潜力。首先,利用LLM搭建的智能体系统可以自动化地批改大量解答题作业,极大地减轻了教师的工作负担;其次,LLM可以通过对学生解题过程的详细建模,提供个性化的反馈,帮助学生更好地理解和改进自己的解题思路、为学生提供更为精准和有针对性的学习指导;LLM还可以进行大规模的数据分析,为教师提供班级整体的错误统计和分析,帮助教师针对性地调整教学策略。通过本研究的探索和实践,希望能为现代教育的变革和发展贡献一份力量,推动教育向更加智能化、个性化和公平化的方向发展。这一研究的探索和实践,不仅丰富了LLM的应用场景,也为教育领域的技术创新提供了新的思路和方法。



\section{相关研究综述}

