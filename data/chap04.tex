% !TeX root = ../__ylc_main.tex

\chapter{系统设计与实现}

本章主要介绍本研究的系统设计和实现。受上一章预实验的启发,我们设计了一套软硬件结合的系统,用来采集学生解题过程的数据并分析,进行学生思维能力评估,最终给出解题分析报告。首先介绍系统的整体架构,然后详细介绍了系统的各个模块的设计和实现,包括数据采集硬件、数据处理、自动批改流程等。

\section{系统架构}

本研究的系统主要由硬件和软件两部分组成。硬件部分主要负责数据采集,包括学生解题过程的视频、音频和笔迹信息。软件部分主要负责数据处理和分析,包括解题过程的思路建模、错因分析等。

\section{数据采集硬件}

本研究使用了安卓平台作为音视频数据采集设备。安卓平台具有广泛的应用场景,可以方便地进行软硬件的开发和定制。具体来讲,我们选用了荣耀平板 V8 Pro 作为数据采集设备。荣耀平板 V8 Pro 是一款性能较为优秀的安卓平板,搭载了天玑 8100 处理器,拥有 8GB 的内存和 128GB 的存储空间,搭载一块 12.1 英寸, 2.5K 像素分辨率, 144Hz刷新率的屏幕。这款平板的性能满足我们的数据采集要求,屏幕显示效果也完全能够胜任给学生展示题目以及交互的需求。我们使用安卓平板电脑设备的摄像头来进行视频和音频信息的采集。

手写笔迹方面,我们使用了 罗博智慧笔 T8C 型号手写板,这是一款性价比较高的数位板,由一块板子和一支压感笔组成,具有高灵敏度、高分辨率、高精度等特点,可以准确地记录学生的笔迹信息;充电后续航在20小时左右,满足我们的采集数据要求。它可以通过蓝牙与手机或者平板电脑连接,也可通过USB或蓝牙连接PC,支持安卓、iOS以及Windows操作系统。

我们在安卓平板上运行我们开发的数据采集软件,通过蓝牙连接数位板进行数据传输,并在安卓平板上进行数据采集与存储。整体的硬件设施如图\ref{fig:experiment_setup}所示。

\section{数据处理}

由于我们的自动批改流程使用大语言模型作为基础,无论是笔迹信息还是音视频信息,都要转化成文本信息才能被我们的系统所使用。因此,我们需要对采集到的数据进行处理,将其转化为文本信息。