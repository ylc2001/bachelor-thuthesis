% !TeX root = ../__ylc_main.tex

% title = {基于解题过程的学生思维能力评估},

% 中英文摘要和关键字

\begin{abstract}
  传统教学中,学生在完成解题之后由老师统一批改,欠缺了对其过程中的思路的跟踪。人工批改作业费时费力,如果要去挖掘每一个学生解题过程的思路细节,更加繁琐和复杂,不可能做到很细致。

  近年来大语言模型(LLM)的发展成熟,为解决这一问题提供了新的思路。LLM已经具有强大的自然语言理解能力,可以自动化地解析和跟踪学生的解题过程。本研究提出了一种基于LLM的学生思维能力评估方法,通过对解题过程中思路的详细建模,从而评估学生的思维能力。在思路建模的基础上,本研究还实现了解题过程的错因分析,对于学生可以准确地指出发生错误的步骤,对于老师可以获得班级学生的错因统计。

  本研究使用安卓平台采集数据,分别使用数位板和设备摄像头收集解题过程的笔迹信息、视频和音频信息。然后在 LLM 的辅助下完成解题过程的思路建模和错因分析。

  本研究是将 LLM 应用于教育领域来进行思维建模的首次尝试,实验结果表明,该方法可以有效地评估学生的思维能力,为自动化题目讲解以及在中学的大规模应用提供了基础。

  % 关键词用“英文逗号”分隔,输出时会自动处理为正确的分隔符
  \thusetup{
    keywords = {思维建模, 大语言模型, 解题过程分析, 人机交互},
  }
\end{abstract}

\begin{abstract*}
  % English abstract, translation of above Chinese abstract

  In traditional teaching, students' detailed thinking in the process of solving problems is rarely considered by teachers. Manual correction of homework is time-consuming and laborious. It is more cumbersome and complex to dig into the details of the thinking process of each student, and it is impossible to be very meticulous doing so.

  In recent years, the development of large language models (LLM) has matured, providing new approaches to solve this problem. LLM already has powerful natural language understanding capabilities and can automatically parse and track students' problem-solving processes. This study proposes a method for evaluating students' thinking ability based on LLM, which evaluates students' thinking ability by modeling the details of their thinking process. Based on the modeling of ideas, this study also implements the analysis of the causes of errors in the problem-solving process, which can accurately point out the steps where students make mistakes and provide teachers with statistics on the causes of errors in the class.

  This study collects data on the Android platform, using a graphics tablet and device camera to collect handwriting information, video, and audio information during the problem-solving process. Then, with the assistance of LLM, the thinking process modeling and error analysis of the problem-solving process are completed.

  This study is the first attempt to apply LLM to the field of education for thinking modeling. The experimental results show that this method can effectively evaluate students' thinking ability, providing a foundation for automated problem explanation and large-scale application in secondary schools.

  % Use comma as separator when inputting
  \thusetup{
    keywords* = {Thinking Modeling, Large Language Model, Problem Solving Process Analysis, Human-Computer Interaction},
  }
\end{abstract*}
