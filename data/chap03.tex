% !TeX root = ../__ylc_main.tex

\chapter{学生思维能力建模方法}

\section{学生表述解题过程的特点}

在解决问题的过程中,人类学生对于解题过程思路的理解与机器对其的理解往往存在差距。学生在解题过程中,会通过一系列的思维活动理解问题、选择解题方法、实施解题过程以及验证结果。这一过程涉及到多种认知活动,包括信息提取、推理、计算和验证等。然而,学生在表述解题过程时,往往会由于多种因素的影响,导致其表述存在一定的不确定性和复杂性。这使得机器在理解学生的解题过程时面临一定的挑战。为了更好地理解学生的解题过程,需要对学生表述解题过程的特点进行详细分析。

对于本研究关注的高中数学解答题,学生在完成数学解题时,通常会有一套较为固定的思维路径和方法。一般来说,学生在解题过程中会经历以下几个阶段:

\begin{itemize}
    \item 理解问题:学生首先需要对题目进行全面的理解,明确题目所要求的目标和已知条件。这一阶段学生会通过阅读、分析题目并提取关键信息来形成对问题的初步认识。
    \item 选择方法:在明确问题之后,学生需要选择适当的解题方法或策略,明确解题路径。
    \item 实施解题过程:学生根据所选的解题方法一步步进行操作,这一过程中通常会涉及到一系列的计算和推理。就逻辑性较强的数学学科的推导而言,对于每一步推理,往往遵循着这样的模式:基于某些已知条件,运用某个特定的数学规律,得到某个新的结论。
\end{itemize}

对于解题过程中每一步的推理,此处进行进一步明确以方便下文的讨论。在解题的初始状态,学生在读题并理解后,脑中会有一个信息的集合,这个集合包括了所有题给条件信息,以及学生记忆中所有与本题相关的知识点和方法。随后的每一步推理,可以认为是在这个信息集合的基础上,运用合适的数学方法,基于已有信息集合的一个子集,通过计算或推导得到一个新的结论,这个新的结论会被加入到信息集合中。这个过程会一直持续到学生得到最终的解答。在这个过程中,学生的思维活动是一个动态的、逐步推理的过程,每一步推理都是基于已有的信息(可能包含先前推出的结果),不断更新本题的“内存空间”。

然而,学生在表述其解题过程时,往往存在一些特点,这些特点使得机器理解学生的解题过程变得复杂。首先,由于人与机器感知方式的差异,二者所能收集到的信息存在差异。其次,人类的记忆是一个动态的、非格式化的、可能出现模糊和错误的信息载体。学生在表述解题过程时,可能会使用不同的语言表达方式和数学符号,不同学生的语言习惯、用词偏好以及对符号的使用存在差异;同时,学生描述某一步推理时的表述方式可能是比较高级的、抽象的(例如“通过观察可以得知”),也可能是详细的、底层的、具体的(例如“根据勾股定理和上一步求出的两条边的长度,可以得到……”)。学生在解题时口头描述的解题过程,是非结构化的语言数据,相较于结构化的书面解答,口头描述包含更多的噪音和不确定性,机器需要具备更强的自然语言处理能力来提取有用信息。此外,学生在解题过程中可能会出现思维跳跃、错误和修正等现象,也有可能出现解题中途发现遇到困难,换一种方法从头开始的情况,这些现象使得学生的解题过程不是一个线性的、单向的过程,而是一个动态的、多变的过程。

这些因素都增加了机器解析人类学生解题过程的难度。比如在知识和已有信息的储存记忆方面,机器如果使用某种数据库来储存高中数学知识的话,一定是有统一的数据结构的、清晰的、无误差的,这与学生记忆所可能出现的模糊性相反。考虑到学生表述解答过程中的多样性,在处理学生的表述时往往面临着机器能力理解不足、从机器规则难以获得预期效果等问题,传统方法也缺乏将学生表述的的非结构化步骤转化为结构化机器数据的能力。

为了弥合机器与人类学生对于做题推理步骤理解的差距,本研究认为需要提出新的有效方法来设计系统,从而对学生的解题过程进行有效的建模。机器需要具备较强的自然语言处理能力,能够解析学生的语言表述和数学符号,并将其转化为可处理的逻辑结构。此外,机器还需要能够处理多步推理和动态修改的复杂性,确保对学生解题思路的准确跟踪和建模。最重要的是,机器需要具备可靠的错误识别和分析能力,能够在学生犯错时识别出错误所在,并进行相应的错因分析,进而实现个性化辅导。

为了解决上述问题,本研究计划基于大语言模型(LLM)来进行学生思维能力的建模。为了探究学生在解题过程中会如何表述自己的思路,本研究先开展了一次预实验,通过让学生口头描述自己的解题过程,收集学生解题过程中的语音、视频和笔迹数据,然后对学生的这些数据进行分析,从中总结学生表述解题的过程中有哪些特点,对解决上述机器理解人类做题推理过程的理解这一挑战有何启示,进而指导后续的系统设计。

\section{解题过程表述特点实验设计}

本研究在着手进行系统设计之前先进行了一次预实验,在实验室环境中请参与者解答高考数学题,通过手写板和智能平板来收集解题过程的笔迹、视频和音频信息,并由人类专家实时观察解题过程。本次实验的详细设计如下所述。

\subsection{实验设计}
在题目选择上,我们选择了三道高考数学题,分别涉及到数列、立体几何和导数的知识,题目难度适中,既有一定的思维含量、需要一定的高中数学知识和技能来求解、可能会有多种解法,难度又没有很高以至于令大多数学生感到无从下手,适合用于本次实验。这三道题目分别是:

\begin{description}
    \item[题目一] 第一道题目的描述。
    \item[题目二] 第二道题目的描述。
    \item[题目三] 第三道题目的描述。
\end{description}

记 $S_n$, 为数列 $\left\{a_n\right\}$ 的前 $n$ 项和, 设甲: $\left\{a_n\right\}$ 为等差数列:乙: $\left\{\frac{S_n}{n}\right\}$ 为等差数列. 则
A. 甲是乙的充分条件但不是必要条件
B. 甲是乙的必要条件但不是充分条件
C. 甲是乙的充要条件
D. 甲既不是乙的充分条件也不是乙的必要条件


已知圆锥的顶点为 $P$, 底面圆心为 $O, A B$ 为底面直径, $\angle A P B=120^{\circ}, P A=2$, 点 $C$ 在底面圆周上, 且二面角 $P-A C-O$ 为 $45^{\circ}$, 则
A. 该圆雉的体积为 $\pi$
B. 该圆雉的侧面积为 $4 \sqrt{3} \pi$
C. $A C=2 \sqrt{2}$
D. $\triangle P A C$ 的面积为 $\sqrt{3}$


已知函数 $f(x)=a\left(\mathrm{e}^x+a\right)-x$.
(1) 讨论 $f(x)$ 的单调性:
(2)证明:当 $a>0$ 时, $f(x)>2 \ln a+\frac{3}{2}$.

\subsubsection{实验流程}